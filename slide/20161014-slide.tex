\documentclass[12pt, unicode]{beamer}
\usetheme{Warsaw}
\usepackage{luatexja}
\usepackage{listings}
\usepackage{listings}
\lstset{
  basicstyle=\ttfamily\bfseries,
  commentstyle=\color{red}\itshape,
  stringstyle=\color{black!30!green},
  showstringspaces=false,
  keywordstyle=\color{blue}\bfseries
}
%color
\definecolor{battleshipgrey}{rgb}{0.52, 0.52, 0.51}

\title{How to apply Rust in Real World?}
\author{Hiroshi Hatake}
\date[2016/03/12]{Technical information sharing seminar}

\begin{document}

\frame{\maketitle}

\begin{frame}{Introduction}
\begin{block}{What is Rust?}
\textbf{Rust} is a systems programming language that runs blazingly fast, prevents segfaults, and guarantees thread safety.
\end{block}
\onslide<1->
\begin{itemize}
\item<2-> zero-cost abstractions
\item<3-> guaranteed memory safety
\item<4-> threads without data races
\item<5-> trait-based generics
\item<6-> pattern matching
\item<7-> type inference
\end{itemize}
\end{frame}

\begin{frame}{How to apply Rust in real world -- Motivation}
  \onslide<1->
  \begin{block}{Motivation}
  \begin{itemize}
  \item<1-> Rust runs fast and guarantee safety.
  \item<2-> And, Rust runs on Windows same as *nix systems.
  \end{itemize}
  \end{block}
\end{frame}

\begin{frame}{An example -- grnenv-rs}
\begin{block}{grnenv -- switch Groonga versions}
grnenv is one of the switching Groonga version tool.
\end{block}
\onslide<1->
\begin{itemize}
\item<2-> It requires bash.
\item<3-> It assumes that Linux environment.
\item<4-> But, I want to use switching tool like this on Windows.
\item<5-> If possible, I want to create \textbf {executables}. Because Windows can handle PE(portable executables) for Windows 2k or older.\footnote[frame]{Perhaps, it can handle executables for Windows 95.}
\end{itemize}
\end{frame}

\begin{frame}{Rust is suitable for this purpose?}
\begin{block}{Rust is suitable for this purpose?}
Halfly yes, halfly not. Because....
\end{block}
\onslide<1->
\begin{itemize}
\item<2-> It requires Visual C++ 2015 Runtime. No more needed.
\item<3-> It can create executables.
\item<4-> But, it is hard to handle \textbf {lifetime}.
\item<5-> Also, It is hard to handle and extend Trait in some cases.
\end{itemize}
\end{frame}

\begin{frame}{Rust has difference type system than ever.}
\begin{block}{Rust can represent abnormal value in type.}
Rust has \textbf {Option} and \textbf{Result} types.
\end{block}
\onslide<1->
\begin{itemize}
\item<2-> \textbf {Option} can contain normal value within ``\textbf{Some}'' and \textbf{None} which represents ``abnormal value'' like as ``NULL''.
\item<3-> \textbf {Result} can contain normal value in ``\textbf{Ok}'' and ``\textbf{Err}'' which can contain error information.
\end{itemize}
\end{frame}

\begin{frame}{What is benefits for these types?}
  \begin{block}{}
    I will show a few questions.
  \end{block}
\end{frame}

\newcommand\Small{\fontsize{9}{9.2}\selectfont}
%% Define highlighting for Rust.
\lstdefinelanguage{Rust} {
  morecomment = [l]{//},
  morecomment = [l]{///},
  morecomment = [s]{/*}{*/},
  morestring=[b]",
  sensitive = true,
  morekeywords = {extern, return, crate, fn, let, use, mut, impl, for, struct, pub, if, type, pub, match}
}
\begin{frame}[fragile]{What is benefits for these types?}
  \begin{block}{}
    Will this function return abnormal value?
  \end{block}
  \begin{lstlisting}[language={Rust},basicstyle=\ttfamily\Small]
    pub fn read_dir<P: AsRef<Path>>(path: P) -> ?
  \end{lstlisting}
\end{frame}

\begin{frame}[fragile]{What is benefits for these types?}
  \begin{block}{}
    First. Will this function return abnormal value?
    \newline
    Yes. It returns \textbf{Result}.
  \end{block}
  \begin{lstlisting}[language={Rust},basicstyle=\ttfamily\Small]
    type Result<T> = Result<T, Error>;
    pub fn read_dir<P: AsRef<Path>>(path: P)
      -> std::io::Result<ReadDir>
  \end{lstlisting}
\end{frame}

\begin{frame}[fragile]{What is benefits for these types?}
  \begin{block}{}
    Second. Will this function return abnormal value?
  \end{block}
  \begin{lstlisting}[language={Rust},basicstyle=\ttfamily\Small]
    // write all buffer into writing target.
    fn write_all(&mut self, buf: &[u8]) -> ?
  \end{lstlisting}
\end{frame}

\begin{frame}[fragile]{What is benefits for these types?}
  \begin{block}{}
    Second. Will this function return abnormal value?
    \newline
    Yes. It returns \textbf{Result}.
  \end{block}
  \begin{lstlisting}[language={Rust},basicstyle=\ttfamily\Small]
    fn write_all(&mut self, buf: &[u8]) -> Result<()>
  \end{lstlisting}
\end{frame}

\begin{frame}[fragile]{What is benefits for these types?}
  \begin{block}{}
    Third. Will this function return abnormal value?
  \end{block}
  \begin{lstlisting}[language={Rust},basicstyle=\ttfamily\Small]
    // obtain user's home directory.
    pub fn home_dir() -> ?
  \end{lstlisting}
\end{frame}

\begin{frame}[fragile]{What is benefits for these types?}
  \begin{block}{}
    Third. Will this function return abnormal value?
    \newline
    Ofcource, Yes!!!

    Because \$HOME always does not exist.
    \newline
    When without sudo -E or using more tighten sudo settings, \$HOME cannot obtain.
  \end{block}
  \begin{lstlisting}[language={Rust},basicstyle=\ttfamily\Small]
    pub fn home_dir() -> Option<PathBuf>
  \end{lstlisting}
\end{frame}

\begin{frame}[fragile]{A person says....}
  \begin{block}{}
    The three laws of disallow NULL.\footnote[frame]{https://twitter.com/gakuzzzz/status/783616563102388224}
  \end{block}
\onslide<1->
\begin{itemize}
\item<2-> Don't receive NULL
\item<3-> Don't return NULL
\item<4-> Don't write NULL
\end{itemize}
\end{frame}

\begin{frame}[fragile]{A person says....}
  \begin{block}{}
    The three laws of disallow NULL.\footnote[frame]{https://twitter.com/gakuzzzz/status/783616563102388224}
    \newline
    For Rust version.
    \onslide<2->
    Rust does not have NULL, Yay!
    \newline
    But Rust has a few danger things like as unwrap().
  \end{block}
\onslide<2->
\begin{itemize}
\item<3-> Don't use unwrap()
\item<4-> Don't dismiss error values
\item<5-> Don't use panic! if it cannot recover
\end{itemize}
\end{frame}

\newcommand\SSmall{\fontsize{7}{7}\selectfont}
\begin{frame}[fragile]{For safety}
  \begin{block}{}
    Rustish guaranting ways.
  \end{block}
\onslide<2->
\begin{itemize}
\item<2-> Don't use unwrap()
\end{itemize}
  \begin{lstlisting}[language={Rust},basicstyle=\ttfamily\SSmall]
    let f = fs::remove_file(shim_dir.join("source-groonga.sh"))
              .unwrap();
  \end{lstlisting}
\end{frame}

\begin{frame}[fragile]{For safety}
  \begin{block}{}
    Rustish guaranting ways.
  \end{block}
\onslide<1->
\begin{itemize}
\item<1-> Don't use unwrap(). Use \textbf{try!}.
\end{itemize}
  \begin{lstlisting}[language={Rust},basicstyle=\ttfamily\SSmall]
    let f = try!(fs::remove_file(shim_dir.join("source-groonga.sh")));
  \end{lstlisting}
\end{frame}

\begin{frame}[fragile]{For safety}
  \begin{block}{}
    Rustish guaranting ways.
  \end{block}
\onslide<1->
\begin{itemize}
\item<1-> Don't dismiss error values
\end{itemize}
  \begin{lstlisting}[language={Rust},basicstyle=\ttfamily\SSmall]
    let _ = f.write_all(&contents.as_bytes());
    f.sync_data()
  \end{lstlisting}
\end{frame}

\begin{frame}[fragile]{For safety}
  \begin{block}{}
    Rustish guaranting ways.
  \end{block}
\onslide<1->
\begin{itemize}
\item<1-> Don't dismiss error values. Use \textbf{pattern matching} to handle error.
\end{itemize}
\begin{lstlisting}[language={Rust},basicstyle=\ttfamily\SSmall]
  match f.write_all(&contents.as_bytes()) {
    Ok(_) => return Ok(()),
    Err(e) => println!("{}", e),
  }
  f.sync_data()
  \end{lstlisting}
\end{frame}

\begin{frame}[fragile]{For safety}
  \begin{block}{}
    Rustish guaranting ways.
  \end{block}
\onslide<1->
\begin{itemize}
\item<1-> Don't use panic! if it cannot recover.
\end{itemize}
  \begin{lstlisting}[language={Rust},basicstyle=\ttfamily\SSmall]
    let home = env::home_dir().unwrap();
  \end{lstlisting}
\end{frame}

\begin{frame}[fragile]{For safety}
  \begin{block}{}
    Rustish guaranting ways.
  \end{block}
\onslide<1->
\begin{itemize}
\item<1-> Don't use panic! if it cannot recover.
\end{itemize}
  \begin{lstlisting}[language={Rust},basicstyle=\ttfamily\SSmall]
    let home = env::home_dir()
                .unwrap_or_else(|| panic!("Cound not found homedir."));
  \end{lstlisting}
\end{frame}

\begin{frame}[fragile]{Someone would think as....}
  \begin{block}{}
    OK. I studied Rustish guaranting ways.
    \newline
    Always using Option and Result for abnormal values makes everything OK.
  \end{block}
\end{frame}

\begin{frame}[fragile]{Someone would think as....}
  \begin{block}{}
    Answer: Sadly, No.
    \newline
    Current Rust does not handle Abstract data type in main() function.
    It can handle only i32(=normally, int) type values, like ..., -2, -1, 0, 1, 2, ....\footnote[frame]{This is intended behaviour. see also: https://github.com/rust-lang/rust/issues/12130\#issuecomment-34583413}
  \end{block}
\end{frame}

\begin{frame}{Conclusion}
  \begin{itemize}
  \item Rust works on Windows same as *nix.
  \item Rust does not have NULL in concept.
  \item Rust has some of danger method like as unwrap().
  \item Using Option and Result is better in most cases.
  \item Some of places is not usable Result type.
  \end{itemize}
\end{frame}

\frame{\centering \Large Happy without NULL life with Rust! Enjoy!!}

\frame{\centering \Large Any questions?}

\end{document}
